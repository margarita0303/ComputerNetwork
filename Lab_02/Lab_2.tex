\documentclass[12pt]{article}

\usepackage{cmap}
\usepackage[T2A]{fontenc}
\usepackage[utf8]{inputenc}
\usepackage[russian]{babel}
\usepackage{graphicx}
\usepackage{amsthm,amsmath,amssymb}
\usepackage[russian,colorlinks=true,urlcolor=red,linkcolor=blue]{hyperref}
\usepackage{enumerate}
\usepackage{datetime}
\usepackage{minted}
\usepackage{fancyhdr}
\usepackage{lastpage}
\usepackage{color}
\usepackage{verbatim}
\usepackage{tikz}

\def\NAME{Задачи по 2-й лабораторной}

\parskip=0em
\parindent=0em

\sloppy
\voffset=-20mm
\textheight=235mm
\hoffset=-25mm
\textwidth=180mm
\headsep=12pt
\footskip=20pt

\pagestyle{empty}

% Основные математические символы
\DeclareSymbolFont{extraup}{U}{zavm}{m}{n}
\DeclareMathSymbol{\heart}{\mathalpha}{extraup}{86}
\newcommand{\N}{\mathbb{N}}   % Natural numbers
\newcommand{\R}{\mathbb{R}}   % Ratio numbers
\newcommand{\Z}{\mathbb{Z}}   % Integer numbers
\def\INF{\t{+}\infty}         % +inf
\def\EPS{\varepsilon}         %
\def\SO{\Rightarrow}          % =>
\def\EQ{\Leftrightarrow}      % <=>
\def\t{\texttt}               % mono font
\def\c#1{{\rm\sc{#1}}}        % font for classes NP, SAT, etc
\def\O{\mathcal{O}}           %
\def\NO{\t{\#}}               % #
\renewcommand{\le}{\leqslant} % <=, beauty
\renewcommand{\ge}{\geqslant} % >=, beauty
\def\XOR{\text{ {\raisebox{-2pt}{\ensuremath{\Hat{}}}} }}
\newcommand{\q}[1]{\langle #1 \rangle}               % <x>
\newcommand\URL[1]{{\footnotesize{\url{#1}}}}        %
\newcommand{\sfrac}[2]{{\scriptstyle\frac{#1}{#2}}}  % Очень маленькая дробь
\newcommand{\mfrac}[2]{{\textstyle\frac{#1}{#2}}}    % Небольшая дробь
\newcommand{\score}[1]{{\bf\color{red}{(#1)}}}

% Отступы
\def\makeparindent{\hspace*{\parindent}}
\def\up{\vspace*{-0.3em}}
\def\down{\vspace*{0.3em}}
\def\LINE{\vspace*{-1em}\noindent \underline{\hbox to 1\textwidth{{ } \hfil{ } \hfil{ } }}}
%\def\up{\vspace*{-\baselineskip}}

%\rhead{Практика \CURNO. \NAME.}
\renewcommand{\headrulewidth}{0.4pt}

\lfoot{}
\cfoot{\thepage\t{/}\pageref*{LastPage}}
\rfoot{}
\renewcommand{\footrulewidth}{0.4pt}

\newenvironment{MyList}[1][4pt]{
  \begin{enumerate}[1.]
  \setlength{\parskip}{0pt}
  \setlength{\itemsep}{#1}
}{       
  \end{enumerate}
}
\newenvironment{InnerMyList}[1][0pt]{
  \vspace*{-0.5em}
  \begin{enumerate}[a)]
  \setlength{\parskip}{#1}
  \setlength{\itemsep}{0pt}
}{
  \end{enumerate}
}

\newcommand{\Section}[1]{
  \refstepcounter{section}
  \addcontentsline{toc}{section}{\arabic{section}. #1} 
  %{\LARGE \bf \arabic{section}. #1} 
  {\LARGE \bf #1} 
  \vspace*{1em}
  \makeparindent\unskip
}
\newcommand{\Subsection}[1]{
  \refstepcounter{subsection}
  \addcontentsline{toc}{subsection}{\arabic{section}.\arabic{subsection}. #1} 
  {\Large \bf \arabic{section}.\arabic{subsection}. #1} 
  \vspace*{0.5em}
  \makeparindent\unskip
}

% Код с правильными отступами
\newenvironment{code}{
  \VerbatimEnvironment

  \vspace*{-0.5em}
  \begin{minted}{c}}{
  \end{minted}
  \vspace*{-0.5em}

}

% Формулы с правильными отступами
\newenvironment{smallformula}{
 
  \vspace*{-0.8em}
}{
  \vspace*{-1.2em}
  
}
\newenvironment{formula}{
 
  \vspace*{-0.4em}
}{
  \vspace*{-0.6em}
  
}

\definecolor{dkgreen}{rgb}{0,0.6,0}
\definecolor{brown}{rgb}{0.5,0.5,0}
\newcommand{\red}[1]{{\color{red}{#1}}}
\newcommand{\dkgreen}[1]{{\color{dkgreen}{#1}}}

\begin{document}

\pagestyle{fancy}

\vspace*{-0.0em}
\Section{\NAME} % Задачи на тему 
\vspace*{-1.0em}

\begin{MyList}[4pt]

  \item[1.] $d_{\text{сквозн}} = \frac{NL}{R}$ - за столько первый пакет доберется до приемника. В этот момент последний пакет будет на $N - (P-1)$ маршрутизаторе, то есть ему останется пройти через $N - (N - (P - 1)) = P - 1$ соединений. Таким образом, задержка для $P$ пакетов составит $\frac{NL}{R} + \frac{(P-1)L}{R}$.
  
  \item[2.] Скорость передачи данных в такой сети будет такой, какая минимальная скорость передачи среди всех соединений, то есть $200$ Кбит/с. Переведем в Кбит, получится, что  $5$ Мб = $40000$ Кбит, поэтому передача займет $\frac{40000}{200} = 200$ секунд.
  
  \item[3.] Вероятность одновременной передачи данных $k$ пользователями составляет \\ $C_{60}^k \cdot (0.2)^k \cdot 0.8^{60 - k}$. Тогда вероятность передачи данных $12$ и более пользователями равна 1 минус вероятность передачи данных менее 12 пользователями, то есть 
  $$ 1 - C_{60}^0 \cdot (0.2)^0 \cdot 0.8^{60 - 0} - ... - C_{60}^{11} \cdot (0.2)^{11} \cdot 0.8^{60 - 11} $$
  
  \item[4.] Общая задержка для $P$ пакетов составит $\frac{NL}{R} + \frac{(P-1)L}{R}$, как мы уже выяснили. $L = s + 80$, $N = 3$, $P = \frac{X}{s}$. Получится формула: 
  $$ d = \frac{3(s + 80)}{R} + \frac{(\frac{X}{s} - 1) (s + 80)}{R} =  \frac{3s + 240 + X + \frac{80X}{s} - s - 80}{R} = \frac{2 s^2 + 160 s + s X + 80 X}{sR} = $$
  $$ = \frac{2s}{R} + \frac{160}{R} + \frac{X}{R} + \frac{80X}{sR} $$
  Найдем производную: 
  $$ d^{'} = \frac{2}{R} - \frac{80X}{s^2 R} $$
  Ее минимум достигается при $s = \sqrt{40X}$, это и будет ответом. 
  
  \item[5a.] Общая задержка, равная сумме задержек ожидания и передачи, при $I < 1$ будет равна: 
  $$ d = \frac{L}{R} + \frac{IL}{R(1 - I)} = \frac{L}{R(1-I)} $$
  
  \item[5б.] Мы вычислили общую задержку как $ \frac{L}{R(1-I)} $, подставим сюда $I = \frac{La}{R}$ и получим $$d = \frac{L}{R} \frac{1}{(1 - \frac{La}{R})} $$
  Должно быть так, что $\frac{L}{R} < \frac{1}{a}$, тогда $\frac{La}{R} < 1$, и формула имеет смысл. Если $\frac{L}{R}$ слишком мало, то общая задержка будет крайне мала. Если $\frac{L}{R}$ не слишком мало, тогда задержка будет существенной. 

	  
\end{MyList}


\end{document}
 
